

\labday{Monday, 29 February 2016}

\experiment{EMC Labs}

\subexperiment{Introduction}

look at the network analyzer which can do a lot of nice things.

the whole system is calibrated to 50 Ohm.

10hz to 500MHz for source, and records all the measurements

measures the input impediance, transfer, and measure the output impediance

slide with different stuff that can be measured.

one of the labs:
ground plane -> serial ground should not be used cause crappy; using loudspeaker cables cause they're cheap
users are represented with capacitors

second lab:
investigate shielding effect of different materials, steel grud, steel, mumetal,brass,coppe,alumonum

one transmitter and one receiver, inbetween the shield -> measure now the atteniuation

the other lab:
look into components , R,L,C, then look at the circuit board:
the greem plate with a lot of strips, different traces



Component analyzing with network analyzer
different components with different values of R, C and L
basically, each component gets crappy at some specific frequency.

and for high frequency applications one does not use resistors.



next experiment:
now use the serial ground config
interesting thing: reducing noise by using capacitors doe not work properly, because they introduce some noise by themselves (???)

next experiment:
use the green plane with different ground configuration, the best one to use is the one thats closest to a coax cable, because there's the least amount of noise introduced into ground by the signal (best attenuation of about -30dB or something



\subexperiment{Summary}
